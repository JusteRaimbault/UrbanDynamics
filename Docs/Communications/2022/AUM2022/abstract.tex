
\documentclass[10pt]{article}

\usepackage[margin=3cm]{geometry}

\title{Coupling heterogeneous urban models}

\author{J. Raimbault$^{1,2,3,4,\ast}$ and M. Batty$^{2}$\medskip\\
$^{1}$ LASTIG, Univ Gustave Eiffel, IGN-ENSG\\
$^{2}$ Center for Advanced Spatial Analysis, University College London\\
$^{3}$ UPS CNRS 3611 Complex Systems Institute Paris Ile-de-France\\
$^{4}$ UMR CNRS 8504 G{\'e}ographie-cit{\'e}s\medskip\\
$^{\ast}$ \texttt{juste.raimbault@ign.fr}
}

\date{}

\pagenumbering{gobble}


\begin{document}

\maketitle

The integration of multiple dimensions of urban systems is essential for their sustainable planning and management. As decision-making tools, simulation models are in that context a useful medium to link together these various aspects. The coupling of heterogeneous urban models, in terms of scale, ontology, data, and implementation among others, is therefore an important methodological hurdle to be dealt with. In this contribution, we summarise some recent use of scientific workflow engines to build modular urban models. In particular, the OpenMOLE workflow engine enables seamless modle embedding - whatever the implementation - and provides a scripting language which is used to couple such heterogeneous models and implement the intermediate tasks, resulting in some sort of meta-model in the workflow engine. We illustrate this by coupling the QUANT spatial interaction model with the SPENSER microsimulation model, for the case of the planned Cambridge-Oxford railway corridor. This coupling allows anticipating the infrastructure impacts within the projected demographic context. The second illustration we develop is the coupling of the Matsim transport model with QUANT and SPENSER to build a four-step multimodal transport model generic to any UK functional area. We show the advantage of using this approach by applying model validation methods provided by the OpenMOLE platform: first by running a Global Sensitivity Analysis to model parameters, and second by applying a genetic calibration algorithm to adjust mode shares to observed commuting flows.

	
\end{document}


