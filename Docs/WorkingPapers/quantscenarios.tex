\documentclass[10pt]{article}
\usepackage[margin=2cm]{geometry}


\title{Scenarios for Quant}


\begin{document}

\maketitle


\section{Sustainability scenarios}

\subsection{Existing approaches}


\cite{spiekermann2018multi} integration of the Irpud into a larger European input/output model and conditioning a more precise raster for impact 
% note: not a multi-scale model, only top-down feedbacks



\cite{reutter2018verkehr} and \cite{schwarze2017stadte}: scenarios evaluated with the Irpud model
\begin{enumerate}
	\item Basis scenario: business as usual
	\item Land use regulations (areas that can be built)
	\item Building construction around stations
	\item Energy efficiency: car sharing, e-mobility, building efficiency, fuel consumption
	\item Constraining policies: street capacity reduction, speed limit reduction, regional toll, parking reduction
	\item Public transport development: tramways, more frequencies, price reduction for urbans
	\item Biking: bike paths network development
	\item Walking: shorter walking paths
	\item Integrated policies (combination)
\end{enumerate}

Best policy is not all combined, but a choice of some.



\subsection{Model}

Dimensions that can be acted upon in the model:

\begin{itemize}
	\item Population distribution: typical patterns of urbanisation? (TOD, sprawl, etc.)	
	\item Employment distribution: idem
	\item Coupled pop-employment (less commuting?) - not really given the structure of the model, can add friction for a transportation mode or some network sections but still assume gravity flows - or use constraints ?
	\item Transportation cost: change the beta of one mode indirectly relates to transportation cost (spurious correlation)
\end{itemize}












\bibliographystyle{apalike}
\bibliography{biblio.bib}

	
\end{document}




